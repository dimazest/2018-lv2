\documentclass{IOS-Book-Article}

\usepackage[square,numbers,sort]{natbib}

\usepackage{booktabs}
\usepackage{multirow}

\usepackage{color}
\usepackage[pdfusetitle]{hyperref}
% http://colorschemedesigner.com/#4o42FfqublRMS
\definecolor{linkcolor}{RGB}{66, 54, 122}
\definecolor{citecolor}{RGB}{84, 141, 100}
\definecolor{urlcolor}{RGB}{168, 70, 67}
\hypersetup{
  colorlinks=true,
  linkcolor=linkcolor,
  citecolor=citecolor,
  urlcolor=urlcolor,
}

\usepackage{soul}
\definecolor{hlcolor}{RGB}{255, 221, 163}
\sethlcolor{hlcolor}
\usepackage{mathptmx}

%\usepackage{times}
%\normalfont
\usepackage[T1]{fontenc}
%\usepackage[mtplusscr,mtbold]{mathtime}
%
\begin{document}
\begin{frontmatter}              % The preamble begins here.

%\pretitle{Pretitle}
\title{LV2}
\runningtitle{LV2}
%\subtitle{Subtitle}

\author[A]{\fnms{Anonymous} \snm{Submission}%
\thanks{Corresponding Author: Book Production Manager, IOS Press, Nieuwe Hemweg 6B,
1013 BG Amsterdam, The Netherlands; E-mail:
bookproduction@iospress.nl.}}
%\author[B]{\fnms{Second} \snm{Author}}
%and
%\author[B]{\fnms{Third} \snm{Author}


%\runningauthor{B.P. Manager et al.}
\address[A]{Confidential Review Copy, Do Not Distribute}
%\address[B]{Short Affiliation of Second Author and Third Author}

%\begin{abstract}
%
%\end{abstract}

\begin{keyword}
Corpus Linguistics \sep Latvian \sep Russian \sep English
\end{keyword}
\end{frontmatter}

\thispagestyle{empty}
\pagestyle{empty}

\section*{Introduction}

This paper presents a multilingual, location-anchored corpus of tweets from Latvia. The main challenge of collecting a socially representative corpus from this country is that several languages are used there: Latvian, the main communication language, Russian, the language of the largest minority in Latvia, and English.

%%%%%%%%%
% Problem
%%%%%%%%%
%There are several challenges of building a tweet collection that adequately represents language use in the country.
%
Building a monolingual Latvian collection could be done by harvesting tweets that contain indicative Latvian words which do not present in other languages, similarly to how it is done for Dutch \cite{sang2013}. However, such an approach is not suitable for tweets in Russian and English, as these languages are widely used outside of Latvia.
%
For the same reason, a TREC-like collection building approach \cite{lin2016overview} of filtering the publicly available stream of tweets by language would not work.
%
A tweet collection based on a curated list of users \cite{SANVICENTE16.465} is effective, but extra care must be taken as tracking personal accounts might be problematic due to privacy concerns.
%
A geo-location based collection produces representative results \cite{milajevs:2017:BUCC}. It is not biased linguistically because it is not based on a list of keywords. It is not biased thematically because it is not based on a list of users. However, a large number of tweets is not geo-located, which makes retrieval incomplete.

%%%%%%%%
% Method
%%%%%%%%
To keep a balance between privacy and objectivity maximizing completeness, this work applies a hybrid approach by combining a geo-location based collection procedure with tracking a curated list of users. To minimize privacy issues, only accounts of organizations and public figures are tracked.

%%%%%%%%%
% Results
%%%%%%%%%

%%%%%%%%%%%%%
% So, what...
%%%%%%%%%%%%%

\section{Data collection}
\label{sec:data-collection}

Over the period from 15 April 2017 to \hl{19 April 2018} the initial set of \hl{1\,685\,048} tweets was collected from the \texttt{POST status/filter} endpoint of the Twitter Streaming API.\footnotemark{}
%
\footnotetext{\url{https://developer.twitter.com/en/docs/tweets/filter-realtime/api-reference/post-statuses-filter}}
%
The \texttt{locations} parameter was set to the bounding box of Riga, the capital of Latvia.\footnotemark{}
%
\footnotetext{The coordinates of the bounding box are \texttt{23.9325829, 56.8570671, 24.3247299, 57.0859184}.}
%
In addition to the location, \hl{420} accounts were tracked. \hl{The whole list of screen names together with user ID is available in the supplement.}

To comply with Twitter's terms of service, on \hl{19 April 2018}, the raw tweet data was redownloaded (or rehydrated) to get rid of deleted tweets. The tweets that originated a retweet were added to the collection. Also, we have noticed a large number of tweets that came from Sweden (probably because of the imprecise \texttt{locations} parameter value), so the tweets which location country code was \texttt{SE} were omitted. This resulted in \hl{1\,133\,812} tweets that formed the final collection presented here.

\section{GLobal analysis}
\label{sec:global-analysis}

\begin{table}
  \centering
  \begin{tabular}{lrllrlrlrlrl}
\toprule
lang & total\_count & total\_share & tracked\_source\_share & \multicolumn{2}{l}{lv} & \multicolumn{2}{l}{ru} & \multicolumn{2}{l}{en} & other\_lang\_count & other\_lang\_share \\
{} & source\_lang\_count & source\_lang\_share & source\_lang\_count & source\_lang\_share & source\_lang\_count & \multicolumn{3}{l}{source\_lang\_share} \\
source\_pretty       &             &             &                      &                   &                   &                   &                   &                   &                   &                  &                  \\
\midrule
Twitter Web Client  &      465845 &       34.1\% &                52.4\% &            386342 &             82.9\% &             14718 &              3.2\% &             38715 &              8.3\% &            26070 &             5.6\% \\
Twitter for Android &      227534 &       16.7\% &                 8.3\% &            153578 &             67.5\% &             22351 &              9.8\% &             34632 &             15.2\% &            16973 &             7.5\% \\
Twitter for iPhone  &      205280 &       15.0\% &                14.0\% &            123388 &             60.1\% &             32796 &             16.0\% &             31829 &             15.5\% &            17267 &             8.4\% \\
TweetDeck           &      104345 &        7.6\% &                91.7\% &            102261 &             98.0\% &                75 &              0.1\% &              1470 &              1.4\% &              539 &             0.5\% \\
TVNET Login         &       57733 &        4.2\% &                96.7\% &             26231 &             45.4\% &             30826 &             53.4\% &                23 &              0.0\% &              653 &             1.1\% \\
dlvr.it             &       45337 &        3.3\% &                98.4\% &             44806 &             98.8\% &               134 &              0.3\% &               129 &              0.3\% &              268 &             0.6\% \\
Facebook            &       36327 &        2.7\% &                95.1\% &             13741 &             37.8\% &             20833 &             57.3\% &               450 &              1.2\% &             1303 &             3.6\% \\
Foursquare          &       30708 &        2.3\% &                 0.0\% &             24300 &             79.1\% &               220 &              0.7\% &              1881 &              6.1\% &             4307 &            14.0\% \\
Instagram           &       24654 &        1.8\% &                 1.7\% &              8801 &             35.7\% &              2450 &              9.9\% &              8164 &             33.1\% &             5239 &            21.3\% \\
SKATIES             &       22798 &        1.7\% &                98.0\% &             22782 &             99.9\% &                 0 &                 0 &                 0 &                 0 &               16 &             0.1\% \\
\bottomrule
\end{tabular}
  
  \caption{Global statistics.}
\end{table}

On average, about \hl{3\,000} tweets were collected per day, which is more than \hl{1\,500} by the geo-location based technique in \cite{milajevs:2017:BUCC}.


\bibliographystyle{unsrtnat}
\bibliography{references,dmilajevs_publications}

\end{document}
