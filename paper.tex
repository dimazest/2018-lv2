\documentclass{IOS-Book-Article}

\usepackage{graphicx}

\usepackage[square,numbers,sort]{natbib}

\usepackage{booktabs}
\usepackage{multirow}

\usepackage{color}
\usepackage[pdfusetitle]{hyperref}
% http://colorschemedesigner.com/#4o42FfqublRMS
\definecolor{linkcolor}{RGB}{66, 54, 122}
\definecolor{citecolor}{RGB}{84, 141, 100}
\definecolor{urlcolor}{RGB}{168, 70, 67}
\hypersetup{
  colorlinks=true,
  linkcolor=linkcolor,
  citecolor=citecolor,
  urlcolor=urlcolor,
}

%\usepackage{soul}
%\definecolor{hlcolor}{RGB}{255, 221, 163}
%\sethlcolor{hlcolor}
\newcommand{\hl}[1]{#1}

\usepackage{mathptmx}

%\usepackage{times}
%\normalfont
\usepackage[T1]{fontenc}
%\usepackage[mtplusscr,mtbold]{mathtime}
%

\newcommand{\sn}[1]{\href{https://twitter.com/#1}{\texttt{@#1}}}

\begin{document}
\begin{frontmatter}              % The preamble begins here.

%\pretitle{Pretitle}
\title{Language Bias in a Multilingual Tweet Corpus}
%\runningtitle{LV2}
%\subtitle{Subtitle}

\author[A]{\fnms{Anonymous} \snm{Submission}%
\thanks{Corresponding Author: Book Production Manager, IOS Press, Nieuwe Hemweg 6B,
1013 BG Amsterdam, The Netherlands; E-mail:
bookproduction@iospress.nl.}}
%\author[B]{\fnms{Second} \snm{Author}}
%and
%\author[B]{\fnms{Third} \snm{Author}


%\runningauthor{B.P. Manager et al.}
\address[A]{Confidential Review Copy, Do Not Distribute}
%\address[B]{Short Affiliation of Second Author and Third Author}

%\begin{abstract}
%
%\end{abstract}

\begin{keyword}
Corpus Linguistics \sep Latvian \sep Russian \sep English
\end{keyword}
\end{frontmatter}

\thispagestyle{empty}
\pagestyle{empty}

\section*{Introduction}

This paper presents a multilingual, location-anchored corpus of tweets from Latvia. The main challenge of collecting a socially representative corpus from this country is that several languages are used there: Latvian, the main communication language, Russian, the language of the largest minority in Latvia, and English.

%%%%%%%%%
% Problem
%%%%%%%%%
Building a monolingual Latvian collection could be done by harvesting tweets that contain indicative Latvian words which do not present in other languages, similarly to how it is done for Dutch \cite{sang2013}. However, such an approach is not suitable for tweets in Russian and English, as these languages are widely used outside of Latvia.
%
For the same reason, a TREC-like collection building approach \cite{lin2016overview} of filtering the publicly available stream of tweets by language would not work.
%
A tweet collection based on a curated list of users \cite{SANVICENTE16.465} is effective, but extra care must be taken as a thematic bias might be introduced.
%
An alternative approach would be to retrieve only geo-located tweets. Such a collection would neither be biased linguistically because it is not based on a list of keywords, nor it would be biased thematically because it is not based on a list of users. The downside is that a large number of tweets are not geo-located, which makes retrieval incomplete.

%%%%%%%%
% Method
%%%%%%%%
To keep a balance between objectivity and completeness, this work applies a hybrid approach by combining a geo-location based collection procedure with tracking a curated list of users, which is based on the accounts of Latvian media outlets, politicians, government institutions and public figures.

Our base assumption is that geo-located tweets are a representative and unbiased sample of tweets from Latvia. Thus, an objective collection should exhibit similar properties as the geo-located one. The main property we study in this work is the language proportion. We hypothesize, that an objective extended collection should have similar language distribution as the initial.

To study the collection in more detail, we manually searched the collection for 9 topics of interest, so we could see whether topics introduce bias.

%%%%%%%%%
% Results
%%%%%%%%%

The analysis shows that---despite our attempt---the extended collection is biased toward content in Latvian: proportionally there is more tweets in Latvian in the extended collection than in geo-located. We observe this phenomena not only in the ``macro level'' (the whole collection), but also at ``micro level'' (across individual topics). In all but one topic, Latvian is more dominant than expected.
% IMS I feel like this paragraph is hard to understand because the notion of "topics" hasn't been really defined yet.  Since this is the introduction, the solution might be to make this statement more general.

%%%%%%%%%%%%%
% So, what...
%%%%%%%%%%%%%

Building an unbiased collection is difficult.  In our case the main reason of the introduced bias, might be that Russian and English content is less news-oriented and much more informal. In other words, while it is common to discuss current news in Latvian by engaging with the media, tweets in Russian and English tend to be personal and friend-oriented. Future work should verify whether this is actually the case.
% IMS is this a hypothesis you intend to test here?  Or is this future work?  Or what?

\section{Data collection}
\label{sec:data-collection}

Over the period from 15 April 2017 to \hl{8 May 2018} the initial set of \hl{1\,708\,236} tweets was collected from the \texttt{POST status/filter} endpoint of the Twitter Streaming API.%\footnotemark{}
%
%\footnotetext{\url{https://developer.twitter.com/en/docs/tweets/filter-realtime/api-reference/post-statuses-filter}}
%
The \texttt{locations} parameter was set to the bounding box of Riga, the capital of Latvia. It is small enought to fit into Twitter API restrictions and covers about 40\% of the popilation of Latvia.%\footnotemark{}
%
%\footnotetext{The coordinates of the bounding box are \texttt{23.9325829, 56.8570671, 24.3247299, 57.0859184}.}
% IMS.2 according to http://worldpopulationreview.com/countries/latvia-population/, the population of Riga is less than half the total population of Latvia.  However, Twitter only allows bounding boxes for location to be so big, so you can't bound the whole country.

In addition to the location, 420 accounts were tracked.%\footnotemark{} %\hl{The whole list of screen names together with user ID is available in the supplement.}
%\footnotetext{In this paper, by ``tracking users'', we mean that their user IDs where passed to the Twitter API to collect tweets that they created, or the tweets that are replies by other users to their tweets. Refer to \url{https://developer.twitter.com/en/docs/tweets/filter-realtime/guides/basic-stream-parameters\#follow} for a complete description. Keep in mind that this is different from the case when a user follows another user.}
%
The accounts are mostly Latvian (that is, coming from Latvia, but not necessary produce content in Latvian) news outlets, politicians, businesses, artists and sport clubs. 

% IMS how did you select those accounts? Can you describe them generically here?  (i.e. "Latvian news outlets, famous actors and politicians, and a selection of Eurovision finalists.")  You should make clear you hand picked them based on background knowledge or if this list was scraped from a public resource, since it reflects effort put into making the collection.


%To comply with Twitter's terms of service, on \hl{8 May 2018}, the raw tweet data was redownloaded to get rid of deleted tweets. The tweets that originated a retweet were added to the collection. Also, we have noticed a large number of tweets that came from Sweden (probably because of the imprecise \texttt{locations} parameter value), so the tweets which location country code was \texttt{SE} were omitted. This resulted in \hl{1\,242\,943} tweets that formed the final collection presented here.
% IMS I think "hydrate" is a jargon term that twarc invented but that isn't widely used.

\section{Twitter users}
\label{sec:global-analysis}

\begin{table}
  \centering
  \tiny
  \begin{tabular}{lrllrlrlrlrl}
\toprule
\multirow{2}{*}{Client} & \multicolumn{3}{c}{Tweets} & \multicolumn{2}{c}{Latvian} & \multicolumn{2}{c}{Russian} & \multicolumn{2}{c}{English} & \multicolumn{2}{c}{Other} \\
\cmidrule(r){2-4} \cmidrule(r){5-6} \cmidrule(r){7-8} \cmidrule(r){9-10} \cmidrule(r){11-12}
{} &  Number & Share & Tracked & Number & Share & Number & Share & Number & Share & Number & Share \\
\midrule
Twitter Web Client  &      409355 &       34.4\% &                52.0\% &            340726 &             83.2\% &             13043 &              3.2\% &             33370 &              8.2\% &            22216 &             5.4\% \\
Twitter for Android &      194359 &       16.4\% &                 8.5\% &            132385 &             68.1\% &             19545 &             10.1\% &             28672 &             14.8\% &            13757 &             7.1\% \\
Twitter for iPhone  &      176795 &       14.9\% &                13.7\% &            105144 &             59.5\% &             29666 &             16.8\% &             26975 &             15.3\% &            15010 &             8.5\% \\
TweetDeck           &       91899 &        7.7\% &                91.6\% &             90098 &             98.0\% &                66 &              0.1\% &              1247 &              1.4\% &              488 &             0.5\% \\
TVNET Login         &       48938 &        4.1\% &                96.7\% &             22681 &             46.3\% &             25684 &             52.5\% &                15 &              0.0\% &              558 &             1.1\% \\
dlvr.it             &       39193 &        3.3\% &                98.4\% &             38713 &             98.8\% &               123 &              0.3\% &               117 &              0.3\% &              240 &             0.6\% \\
Facebook            &       31166 &        2.6\% &                95.1\% &             11899 &             38.2\% &             17711 &             56.8\% &               424 &              1.4\% &             1132 &             3.6\% \\
Foursquare          &       28991 &        2.4\% &                 0.0\% &             22762 &             78.5\% &               208 &              0.7\% &              1978 &              6.8\% &             4043 &            13.9\% \\
Instagram           &       23046 &        1.9\% &                 1.8\% &              8347 &             36.2\% &              2242 &              9.7\% &              7591 &             32.9\% &             4866 &            21.1\% \\
SKATIES             &       19015 &        1.6\% &                98.2\% &             19003 &             99.9\% &                 0 &                 0 &                 0 &                 0 &               12 &             0.1\% \\
\bottomrule
\end{tabular}
  
  \caption{Global statistics. ``By followed'' is the percentage of tweets that are created by a followed account, this excludes retweets and replies by not followed accounts.}
\label{tab:source_counts}
\end{table}

On average, about \hl{3\,200} tweets were collected per day, which is more than 1\,500 by the geo-location based technique in \cite{milajevs:2017:BUCC}. Most of the tweets came from the official Twitter clients for Web, Android or iPhone. Together these clients contributed more than 60\% of all collected tweets, refer to Table~\ref{tab:source_counts} for more details.

The twitter web client is used by both the general public and media companies. The top 5 of most active accounts consists only of tracked Latvian media accounts: \sn{DienaLV}, a newspaper, \sn{LA\_lv}, another newspaper, \sn{JaunsLV}, a media portal, \sn{TV3\_Play}, a TV channel, and \sn{dblv}, a newspaper.

Notably, for mobile clients, the majority of tweets comes from the general public, because they are either geo-located or mention a tracked account. The number of tweets coming from tracked accounts, that are mostly media, is lower for Android \hl{(8.4\%)} than for iPhone \hl{(13.8\%)}. The difference can be explained by the fact, that all top 5 Android users are personal accounts, while for iPhone there are only 2 personal accounts in the top 5. \sn{TV3zinas}, a news account of a TV channel, tweets the most from iPhone. \sn{Lattelecom}, a telecommunication company, is the second most active iPhone user. \sn{AstrologiLv} writes about astrology and is the fourth most active iPhone user. All three accounts tweet exclusively in Latvian and were tracked during corpus collection.

%Client applications that produce tweets mostly from tracked accounts are used by the media. 
% IMS I don't understand the previous sentence
%For the TweetDeck, TVNET Login, dlvr.it and SKATIES client applications the top 5 users are tracked media accounts that write dominantly in Latvian. The exceptions are a business account \sn{Kompresori} that was not tracked and tweets in three languages: Latvian, Russian and English, \sn{RusApollo} and \sn{TVNET\_rus} who tweet identical content exclusively in Russian, and \sn{SejasLV}, an exclusively Latvian account writing about celebrities, which was not tracked.

% IMS terminology jumble. You are tracking, and also other users are tracking, and you selected users who were tracking (well, actually, probably retweeting) tweets from accounts you were tracking already.  You want this section to read like a recipe for your reader who wants to duplicate your effort in Swahili or whatever.

%TweetDeck is used by \sn{DelfiLV}, a major media portal, \sn{LV\_Portals}, an account run by the official government gazette of the Republic of Latvia, \sn{lsmlv}, an account of a publicly funded radio and television organization (LSM), \sn{ltvzinas}, the news account of Latvian Television, \sn{RietumuRadio}, a radio station. 

%TVNET is a media company that controls several media portals. The most active accounts that use their application are \sn{TVnet\_portals}, \sn{RusApollo}, \sn{TVnet\_rus}, \sn{SportsTVNET}, and \sn{SejasLV}.

%The tweet delivery service \texttt{dlvr.it} is used by the newspaper \sn{nralv} and their sport portal \sn{Sportacentrs}, the photo account of the national news agency (LETA) \sn{letafoto}, \sn{Kompresori} and the sport account of Latvian Television \sn{ltvsports}.

%SKATIES delivers video content from several sources. The most active accounts are \sn{skatieslatvija}, \sn{lnt\_lv}, \sn{tv3lv}, \sn{BezTabuTV3} and \sn{TV3Zinas}.

Some tweets come from social media networks. Facebook content mostly comes from tracked media accounts. The top 5 most active users are: \sn{mixnews\_lv}, the Russian edition of a media portal, \sn{Otkrito}, also the Russian edition of \sn{JaunsLV}, \sn{labdienlv}, \sn{ekonimikalv} and \sn{nozare}, all three belonging to one Latvian media portal. The fact that Russian accounts are not active among other clients might suggest that Russian content is not spread initially via Twitter, but is cross-posted from Facebook.
% IMS but you said you were tracking Facebook.  I think you mean that you can only track FB when the content is also posted to Twitter, but not otherwise.

The content that comes from Foursquare and Instagram is created by the general public.
% IMS.2 same comment as above... when you say "comes from" you mean crossposted content, right?  It's confusing because you also say "comes from" to mean the organization authoring the content.

There is a clear pattern that mobile clients are used by personal accounts, while media companies set up a special solution to deliver tweets, probably, integrating it with their content delivery systems or simply using a web client from an office machine.

% IMS I wonder if a flow diagram would illustrate the paths this information takes before it gets on Twitter.

\section{Language}
\label{sec:language}

\begin{figure}
\centering

\caption{Tweet volume by day per language averaged over a rolling window of 7 days.}

\includegraphics[width=\textwidth]{supplement/rehydrated_tweets_count_by_day.pdf}

\label{fig:rehydrated-tweets-count-by-day}
\end{figure}

%%% Local Variables:
%%% mode: latex
%%% TeX-master: "../paper.tex"
%%% End:

Latvian is the dominant language: \hl{917\,259 (73.8\%)} tweets are written in it.\footnotemark{} There are \hl{130\,825 (10.5\%)} tweets in Russian, \hl{122\,490 (9.9\%)} in English and \hl{72\,369\, (6.2\%)} in other languages. This distribution is very different from a geo-located method used in \cite{milajevs:2017:BUCC}, where the distribution is 44.5\% Latvian, 33.9\% Russian and 20.7\% English. Figure~\ref{fig:rehydrated-tweets-count-by-day} shows the tweet volume over time.

\footnotetext{Twitter labels tweets with the language they are written in.}

Latvian dominates almost in every client application. Tweets coming from Foursquare are labeled as Latvian in \hl{78.4\%} cases (see Table~\ref{tab:source_counts}), that might be due to Latvian location names.

Russian tweets constitute of more then half tweets coming from TVNET, where its share is \hl{52.8\%} while Latvian is \hl{46.1\%}, and from Facebook, where the two most active users are Russian media.

A lot of English content is coming form Instagram (\hl{36.1\%}), that might be due to the dominance of hashtags over other text. Also, there are many tweets labeled as written in other languages that come from Foursquare and Instagram. As Instagram posts are more likely to contain hashtags and Foursqure tweets contain location names, tweets should not be assigned a single language label, but instead be labeled as written in several languages.

% IMS the point of the paper is that you believe this is a more representative sample than would be obtained by simpler methods.  Why do you believe it?  Is the hypothesis falsifiable?

% IMS is there code-switching?  How are you deciding that a tweet is in one language vs another?

\section{Topics}
\label{sec:topics}

To investigate the collection further nine topics of interests were defined. The collection was manually searched using keywords to get relevant tweets. The topics are:
\begin{itemize}
\item \texttt{LV001: Winter} Everything about winter.
\item \texttt{LV002: 
The Chronicles of Melanie} Mel\=anijas hronika (The Chronicles of Melanie) is a Latvian movie that was selected for Foreign-Language Category for Oscars.
\item \texttt{LV003: Refugees} A broad topic about refugee crisis in Europe, attitude to immigration and immigrants.
\item \texttt{LV004: Blizzard of Souls} A Latvian movie Dv\=eese\c{l}u putenis (Blizzard of Souls) that is currently in production.
\item \texttt{LV005: Ice hockey} Ice hockey related tweets.
\item \texttt{LV006: Christmas} Christmas.
\item \texttt{LV007: New Year} New Year.
\item \texttt{LV008: Olympics} Tweets about the Olympic games and the Latvian team.
\item \texttt{LV009: Brexit} Brexit.
\end{itemize}

% IMS I note your topics are titled either in Latvian or English.  Is that intentional?  Does it reflect a goal of boosting content from those languages?  Bigger question: why these topics?  More esoteric question: what value are the topics adding to your collection?  How many topics should you have?
Topics exhibit various temporal patterns, refer to Figure~\ref{fig:topic_timeline}. The volume of tweets is constant for long-lasting news stories such as the refugee crisis, Brexit and ice hockey.
%
During the Ice Hockey World Championship the volume of hockey related tweets increases. Similarly, event-based topics---such as Christmas, New Year and Olympics---are the most active during corresponding events, the volume of tweets slowly builds up as an event approaches.

In case of Christmas, we see unexpected activity in March which is due to the discussion of making the Orthodox Christmas a public holiday in Parliament. This topic exhibits some cultural differences and language use. Tweets in Latvian peak during Advent and Christmas in December, tweets in Russian reach maximum both in December and early January when Orthodox Christmas are celebrated. It worth noting that during the discussion whether Orthodox Christmas should be a public holiday, it sometimes was referred as ``Russian Christmas.''

Topics about movies are the smallest volume-wise. For a movie that was shown at several international festivals, we see multilingual content, while tweets about a movie that is still in production are solely monolingual.

The topic about Winter is an example of a seasonal topic, which is mostly about the weather, the pictures of snow and driving conditions.

Table~\ref{tab:topic_lang_counts} shows the total number of tweets per topic and language share for Latvian, Russian and English. For all topics the share of Latvian tweets is higher than on average in the collection (Section~\ref{sec:language}). For Russian and English these numbers are generally lower than global average.

The most notable exception being Russian tweets about New Year, probably because of the content written by Russian tourists. The topic about Winter comes close for Russian with respect to the expected average, but is low for English tweets. Interestingly, for the international news topics, there is proportionally more English content about Brexit than about refugees.

\begin{table}
  \centering

  \begin{tabular}{llrrrr}
\toprule
TopicID  & Title &  Latvian & Russian & English &  Tweets \\
\midrule
LV001    &  85.9\% &  9.6\% &  4.4\% &   9721 \\
LV002    &  88.8\% &  0.9\% & 10.2\% &    215 \\
LV003    &  88.9\% &  6.7\% &  4.4\% &   2968 \\
LV004    & 100.0\% &  0.0\% &  0.0\% &    225 \\
LV005    &  94.5\% &  2.4\% &  3.1\% &  26838 \\
LV006    &  86.5\% &  4.6\% &  8.9\% &   3632 \\
LV007    &  77.2\% & 14.4\% &  8.5\% &   1463 \\
LV008    &  95.3\% &  4.0\% &  0.7\% &   7518 \\
LV009    &  84.7\% &  4.9\% & 10.4\% &   2062 \\
LV010    &  85.5\% & 11.6\% &  2.9\% &   1290 \\
LV011    &  84.0\% & 13.9\% &  2.2\% &   1154 \\
LV012    &  94.6\% &  3.4\% &  2.0\% &    496 \\
LV013    &  94.0\% &  6.0\% &  0.0\% &    783 \\
LV014    &  89.1\% &  1.8\% &  9.1\% &   1022 \\
LV015    &  98.1\% &  1.9\% &  0.0\% &   1200 \\
LV016    &  86.5\% &  1.4\% & 12.1\% &   6804 \\
\bottomrule
\end{tabular}

  
  \caption{Number of relevant tweets per topic and language distribution.}
  \label{tab:topic_lang_counts}
\end{table}

\section{Conclusion and future work}
\label{sec:conclusion}

% IMS The conclusion does not address any of the questions raised in the abstract/introduction.

This paper presents a multilingual tweet collection with some analysis of users, language use and content. The analysis revealed differences in language use between the geo-located collection, global collection and topical sub-collections.

However, the question of whether the collection is any good remains open. It would be easy to test its extrinsic properties, for example, whether it leads to improvements in a language identification system when used as training data. But does not reveal its intrinsic properties.

Riga was chosen as the basis fro the geo-located collection, but the demographic in the city (larger proportion of ethinc Russians)  is different than in the rest of the country, especially its Western parts (larger proportion of ethnic Latvians). Taking this in the account, our global collection might be less biased that the geo-located one! However, it is an open question, whether geo-location of a larger area, for example, by tracking top 10 largest cities, improves the result.

Is it possible to control for bias using the tools described here? If we tracked more accounts that tweet dominantly in English, would that introduce other biases? Is it possible to select a mixture of topics such that every language would have few topics it dominates? How these topics should be explored?

%The challenges of building representative and complete tweet collection are:
%\begin{itemize}
%\item How the content of users who do not engage with media accounts (newspapers, radio stations, etc.~\ldots) might be collected respecting their privacy and not introducing bias into the collection?
%\item How to get hints of what topics are more popular among ``minorities''? What are tweets in Russian and English are about?
%\item Some tweets are multilingual. How can code-switching be detected reliably?
%\end{itemize}

\begin{figure}
  \centering
  \includegraphics[width=0.85\textwidth]{supplement/topic_timeline.pdf}
  \caption{Topic timeline. Note the logarithmic Y-scale, which is the number of tweets.}
  \label{fig:topic_timeline}
\end{figure}


\bibliographystyle{unsrtnat}
\bibliography{references,dmilajevs_publications}

\end{document}
